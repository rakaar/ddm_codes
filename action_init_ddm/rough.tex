% - context ---
The evidence accumulated in time "dt" is adds to the existing decision variable x(t).
x(t+dt) = x(t) + e(dt)

x(t + dt) - x(t) = e(dt)

The above equation can be written as diffusion equation

dx/dt = \mu + \sigma \eta(t)

\mu is the mean evidence accumulated in time dt
\sigma is the standard deviation of the evidence accumulated in time dt

so firstly we need to find \mu and \sigma


% ---- context

% ---- mu and sigma ---

\subsubsection{find mean and variance of the evidence accumulated in time dt}

\paragraph{Mean of the evidence accumulated in time dt}
Let $q_e$ is the quantum evidence accumulated by 1 spike. The evidence is difference between left and right
channels.

$e(dt) =  q_e \cdot (\sum_{i=1}^{n} a_R_i(dt)  - \sum_{i=1}^{n} a_L_i(dt)) $

where $a_R_i$ is the sum of number of spikes in right channel in time dt
And we want to count total number of spikes by adding each Neuron's spikes in time "dt"

Each neuron spikes according to poisson process with mean rate "r". Number of spikes in time "dt" is $r*dt$
Mean Evidence accumulated by single neuron is

$E(a_R_i) = q_e \cdot r \cdot dt$

Assuming all neuron's spiking is independent. Mean evidence of all neurons in right channel is

$E(\sum_{i=1}^{n} a_R_i) = E(q_e \cdot a_R_1) + E(q_e \cdot a_R_2) + ... + E(q_e \cdot a_R_n) 
                         = (q_e \cdot r_R \cdot dt) + (q_e \cdot r_R \cdot dt) ..... (q_e \cdot r_R \cdot dt)
                         = N \cdot q_e \cdot r_R \cdot dt$

Similarly, Mean evidence of all neurons in left channel is

$E(\sum_{i=1}^{n} a_L_i) = N \cdot q_e \cdot r_L \cdot dt$

So, Mean evidence accumulated in time dt is

$E(e(dt)) = q_e \cdot E(\sum_{i}^{N} a_R_i(t) - \sum_{i}^{N} a_L_i(t))

Because mean of Difference Independent Random Variables is difference of their means
          = q_e \cdot E(\sum_{i}^{N} a_R_i(t)) - E(\sum_{i}^{N} a_L_i(t))
          = q_e \cdot N \cdot (r_R - r_L) \cdot dt


So, mean of evidence accumulated in time dt is
E(e(dt)) = \mu dt = q_e \cdot N \cdot (r_R - r_L) \cdot dt

\paragraph{Variance of the evidence accumulated in time dt}

Var(e(dt)) = Var(q_e \cdot (\sum_{i=1}^{n} a_R_i(dt)  - \sum_{i=1}^{n} a_L_i(dt)))
           = q_e^2 Var(\sum_{i=1}^{n} a_R_i(dt)  - \sum_{i=1}^{n} a_L_i(dt))

Variance of difference of 2 Independent Random variables is sum of their variances
           = q_e^2 (Var(\sum_{i=1}^{n} a_R_i(dt)) + Var(\sum_{i=1}^{n} a_L_i(dt)))

In a poisson process, variance is equal to mean
            = q_e^2 (E(\sum_{i=1}^{n} a_R_i(dt)) + E(\sum_{i=1}^{n} a_L_i(dt)))
            = q_e^2 (N \cdot q_e \cdot r_R \cdot dt + N \cdot q_e \cdot r_L \cdot dt)
            = q_e^2 \cdot N \cdot q_e \cdot (r_R + r_L) \cdot dt


So, variance of evidence accumulated in time dt is
Var(e(dt)) = \sigma^2 dt = q_e^2 \cdot N \cdot q_e \cdot (r_R + r_L) \cdot dt

We have mean and variance of evidence in terms of rates. But all we know about stimulus is Sound levels of left and right sounds.
 We need to express rates of right and left channels in terms of sound levels


% ---- mu and sigma ---
subsubsection{Expressing rates in terms of sound levels}




% --- variance ---

\paragraph{Variance}

Remeber variance was defined as
Var(e(dt)) = q_e^2 \cdot N \cdot q_e \cdot (r_R + r_L) \cdot dt

So, basically, same as variance except for q_E^2 instead of q_e and r_R + r_L instead of r_R - r_L
Doing the same substitution as in mean, we get cosh instead of sinh

Var(e(dt)) =   \frac{2q_e^2}{T_0}  10^{\lambda \frac{ABL}{20}} \cosh\left(\lambda \frac{ILD}{X}\right) dt

% ---

Since
 <e(dt)> is the evidence accumulated in each time step, which is \mu dt

 <e(dt)> = \mu dt

  similarly, Variance at each step is (\sigma dB)^2 = \sigma^2 dt

  
So,  Var(e(dt)) = \sigma^2 dt

  we have 

  % put these 2 equations in a box to highllight
  \mu =    \frac{2q_e}{T_0}  10^{\lambda \frac{ABL}{20}} \sinh\left(\lambda \frac{ILD}{X}\right) 
 \sigma^2 =   \frac{2q_e^2}{T_0}  10^{\lambda \frac{ABL}{20}} \cosh\left(\lambda \frac{ILD}{X}\right)

 % ----

 \subsubsection{Things are small}

 \lambda < 1 

 Max value of ILD is 6

 X = 17.3

    So, \lambda \frac{ILD}{X} << 1


And when \phi is small, \sinh(\phi) = \phi and \cosh(\phi) = 1
So accordingly, \mu and \sigma^2 can be approximated as 

\mu =    \frac{2q_e}{T_0}  10^{\lambda \frac{ABL}{20}} \lambda \frac{ILD}{X} 
 \sigma^2 =   \frac{2q_e^2}{T_0}  10^{\lambda \frac{ABL}{20}}


 % ---
 \subsubsection{Putting mean and variance in the diffusion equation}

 the orginal differential equaton with "x" as decision variable is
 
 dx/dt = \mu + \sigma \eta(t)

\begin{equation}
    dx = \mu dt + \sigma \eta(t) dt
    \label{eq:og_diff}
\end{equation}

 Normalization decision variable with "distance between starting point to bound" \theta
 z = x/\theta

 dz = dx/\theta

 \begin{equation}
    dx = \theta dz
    \label{eq:dxdz}
\end{equation}




 time is normalized with average time taken when velocity is zero, t_{\theta} = \frac{\theta^2}{\sigma^2}

 \tau = t/t_{\theta}

 d\tau = dt/t_{\theta}

\begin{equation}
    dt = t_{\theta} d\tau
    \label{eq:dtdtau}
\end{equation}

Putting \ref{eq:dxdz} and \ref{eq:dtdtau} in \ref{eq:og_diff}

\begin{equation}
    \theta dz &= \mu t_{\theta} d\tau + \sigma \eta(t) t_{\theta} d\tau
           dz  &= \frac{1}{\theta} \mu \frac{\theta^2}{\sigma^2} d\tau + \frac{1}{\theta} \sigma \eta(t) \frac{\theta^2}{\sigma^2} d\tau
              dz  &= \mu \frac{\theta}{\sigma^2} d\tau + \frac{\theta}{\sigma} \eta(t) d\tau
              dz  &= \mu \frac{\theta}{\sigma^2} d\tau + \sqrt{t_{\theta}} \eta(t) d\tau
              
\end{equation}

rewriting the above equation
\begin{equation}
    \frac{dz}{d\tau} = \mu \frac{\theta}{\sigma^2} + \sqrt{t_{\theta}} \eta(t)
    \label{eq:dz_by_dtau}
\end{equation}

We already calculated \mu and \sigma^2 in terms of sound levels. So, we can substitute them in above equation

\frac{\mu}{\sigma^2} = \frac{1}{q_e} \tanh(\lambda \frac{ILD}{X})

So, \mu \frac{\theta}{\sigma^2} = \frac{\theta}{q_e} \tanh(\lambda \frac{ILD}{X})
Let \frac{\theta}{q_e} = \theta_e

and somehow \sqrt{\t_theta} \eta(t) = \eta(\tau)
So, \ref{eq:dz_by_dtau} can be written as

\begin{equation}
    \frac{dz}{d\tau} = \theta_e \tanh(\lambda \frac{ILD}{X}) + \eta(\tau)
    \label{eq:dz_by_dtau_final}
\end{equation}


So, whatever be the ABL, when you normalize time and DV, the mean evidence accumulated depends on ILD only.
In terms of TIED,  evideence accumulated depends on ILD as per weber's law(fraction of stimuli) and the RT distributions
are time-scaled versions of each other by a factor of $t_{\theta}$

\begin{equation}
 t_{\theta} &=  \theta^2/\sigma^2
            &= \theta^2/(\frac{2q_e^2}{T_0}  10^{\lambda \frac{ABL}{20}} \cosh(\frac{\lambda \cdot ILD}{X}))
            
    Putting \theta/q_e = \theta_e
            &= \frac{(\theta_e^2 \cdot T_0 \cdot 10^{\lambda \frac{ABL}{20}})}{cosh(\frac{\lambda \cdot ILD}{X})} 
\end{equation}




